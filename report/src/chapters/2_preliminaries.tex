\section{Cơ sở lí thuyết}
\label{sec:preliminaries}

Trong toàn bộ bài báo, ta giả định rằng $C$ là một tập khác rỗng, đóng và lồi trong $\mathbb{R}^{m}$, $f:\mathbb{R}^{m}\rightarrow \mathbb{R}$ là một hàm khả vi trên một tập mở chứa $C$, ánh xạ $\nabla f$ là  liên tục Lipschitz, tức là tồn tại một hằng số $L>0$ sao cho $||\nabla f(x)-\nabla f(y)||\le L||x-y||$ với mọi $x$, $y\in C$. Ta xem xét bài toán tối ưu hóa:
\begin{equation} \label{eq:OP}
    \min_{x\in C}f(x). \tag{OP(f,C)}
\end{equation}

Giả sử rằng tập nghiệm của $(OP(f,C))$ là không rỗng. Trước hết, ta nhắc lại một số định nghĩa và kết quả cơ bản sẽ được sử dụng trong phần tiếp theo của bài báo. Độc giả quan tâm có thể tham khảo Bauschke và Combettes \cite{bauschke2011convex} và Rockafellar \cite{rockafellar1970convex} để biết thêm về các tài liệu tham khảo. Với $x\in\mathbb{R}^{m}$ ký hiệu bởi $P_{C}(x)$ là hình chiếu của $x$ lên $C$, tức là:
\begin{equation}
    P_{C}(x):=\text{argmin}\{||z-x||:z\in C\}.
\end{equation}

\begin{proposition}[Bauschke và Combettes \cite{bauschke2011convex}] \label{prop:1}
    cho rằng
    \begin{enumerate}
        \item[(i)] $||P_{C}(x)-P_{C}(y)||\le||x-y||$ với mọi $x$, $y\in\mathbb{R}^{m}$,
        \item[(ii)] $\langle y-P_{C}(x)$, $x-P_{C}(x)\rangle\le0$ với mọi $x\in\mathbb{R}^{m}$, $y\in C$.
    \end{enumerate}
\end{proposition}

\begin{definition}[Mangasarian \cite{mangasarian1965pseudoconvex}] \label{def:1}
    Hàm $f:\mathbb{R}^{m}\rightarrow\mathbb{R}$ được gọi là:
    \begin{itemize}
        \item lồi trên $C$ nếu với mọi $x$, $y\in C$, $\lambda\in[0,1]$, thỏa mãn rằng
        \[ f(\lambda x+(1-\lambda)y)\le\lambda f(x)+(1-\lambda)f(y). \]
        \item giả lồi trên $C$ nếu với mọi $x$, $y\in C$, thỏa mãn rằng
        \[ \langle\nabla f(x),y-x\rangle\ge0\Rightarrow f(y)\ge f(x). \]
        \item tựa lồi trên $C$ nếu với mọi $x$, $y\in C$, $\lambda\in[0;1]$, thỏa mãn rằng
        \[ f(\lambda x+(1-\lambda)y)\le \max\{f(x);f(y)\}. \]
    \end{itemize}
\end{definition}

\begin{proposition}[Dennis và Schnabel \cite{dennis1983numerical}] \label{prop:2}
    Hàm khả vi $f$ là tựa lồi trên $C$ khi và chỉ khi
    \begin{equation}
        f(y)\le f(x)\Rightarrow\langle\nabla f(x),y-x\rangle\le0.
    \end{equation}
\end{proposition}

Đáng chú ý là ``$f$ là lồi'' $\Rightarrow$ ``$f$ là giả lồi'' $\Rightarrow$ ``$f$ là tựa lồi'' (xem Mangasarian \cite{mangasarian1965pseudoconvex}).

\begin{proposition}[Dennis và Schnabel \cite{dennis1983numerical}] \label{prop:3}
    Giả sử rằng $\nabla f$ là $L$-Lipschitz liên tục trên $C$. Với mọi $x$, $y\in C$, ta có
    \begin{equation} \label{eq:lipschitz}
        |f(y)-f(x)-\langle\nabla f(x),y-x\rangle|\le\frac{L}{2}||y-x||^{2}.
    \end{equation}
\end{proposition}

\begin{lemma}[Xu \cite{xu2002iterative}] \label{lemma:1}
    Cho $\{a_k\}$; $\{b_{k}\}\subset(0;\infty)$ là các dãy sao cho
    $a_{k+1}\le a_{k}+b_{k}\forall k\ge0;$ $\sum_{k=0}^{\infty}b_{k}<\infty.$
    Khi đó, tồn tại giới hạn $\lim_{k\rightarrow\infty}a_{k}=c\in\mathbb{R}$.
\end{lemma}